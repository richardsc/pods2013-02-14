% beamer template plain

\documentclass{beamer}

\setbeamertemplate{navigation symbols}{}

\mode<presentation>
{
  \usetheme{Boadilla}
  % or ...

  %\setbeamercovered{transparent}
  % or whatever (possibly just delete it)
}

\usepackage{color}
\usepackage[english]{babel}
\usepackage[latin1]{inputenc}
\usepackage{times}
\usepackage[T1]{fontenc}
\usepackage{natbib}
\def\newblock{\hskip .11em plus .33em minus .07em}
\usepackage{fancybox}
\usepackage{verbatim}
% Or whatever. Note that the encoding and the font should match. If T1
% does not look nice, try deleting the line with the fontenc.

\newcommand{\pic}[2][1]{\begin{frame}
    \begin{center}
      \includegraphics[width=#1\hsize]{\dir/#2}
    \end{center}
  \end{frame}
}

\title[POD] % (optional, use only with long paper titles)
{An introduction to R for oceanographic analysis}

\subtitle
{(You don't just have to use Matlab for everything)}

% \author
% {author}
\author[Richards]
{Clark Richards}
% - Use the \inst{?} command only if the authors have different
%   affiliation.

% \institute[abbrev] % (optional, but mostly needed)
% {Institute}

\date% (optional)
{2013-02-14}

\subject{Talks}
% This is only inserted into the PDF information catalog. Can be left
% out. 



% If you have a file called "university-logo-filename.xxx", where xxx
% is a graphic format that can be processed by latex or pdflatex,
% resp., then you can add a logo as follows:

% \pgfdeclareimage[height=0.5cm]{university-logo}{dal_logo.pdf}
% \logo{\pgfuseimage{university-logo}}

% Delete this, if you do not want the table of contents to pop up at
% the beginning of each subsection:
% \AtBeginSubsection[]
% {
%   \begin{frame}<beamer>{Outline}
%     \tableofcontents[currentsection,currentsubsection]
%   \end{frame}
% }


% If you wish to uncover everything in a step-wise fashion, uncomment
% the following command: 

%\beamerdefaultoverlayspecification{<+->}

\synctex=1

\usepackage{Sweave}
\begin{document}

\begin{frame}
  \titlepage
\end{frame}

% \begin{frame}{Outline}
%   \tableofcontents
%   % You might wish to add the option [pausesections]
% \end{frame}


\section{Why R?}

\begin{frame}{Why R?}

  \begin{block}{}
    \begin{itemize}
    \item Free (as in beer)!
    \item Open source
    \item Powerful (especially with statistics)
    \item Good package management (and help!)
    \item ...
    \end{itemize}
  \end{block}

\end{frame}

\section{Basics}

\begin{frame}[fragile]{Basics}{Assignment}
  
  R generally uses \verb|<-| instead of \verb|=| to {\it assign}
  variables (\verb|=| is reserved for function arguments and comparisons).
\begin{Schunk}
\begin{Sinput}
> x <- 5
> y <- 'hello world'
> z <- c(1:2, 5, 'something')
> x == 5
\end{Sinput}
\begin{Soutput}
[1] TRUE
\end{Soutput}
\end{Schunk}
  
\end{frame}

\begin{frame}[fragile]{Basics}{Object orientation}

  \begin{itemize}
  \item Objects in R can be assigned a class, with specific functions
    to handle various classes (e.g. summaries, plotting, data
    manipulation)
  \item This makes working with specific classes (i.e. data types)
    quick and easy to code.
  \end{itemize}
\begin{Schunk}
\begin{Sinput}
> class(x)
\end{Sinput}
\begin{Soutput}
[1] "numeric"
\end{Soutput}
\begin{Sinput}
> data(co2)
> class(co2)
\end{Sinput}
\begin{Soutput}
[1] "ts"
\end{Soutput}
\begin{Sinput}
> summary(co2)
\end{Sinput}
\begin{Soutput}
   Min. 1st Qu.  Median    Mean 3rd Qu.    Max. 
  313.2   323.5   335.2   337.1   350.3   366.8 
\end{Soutput}
\end{Schunk}

\end{frame}

\begin{frame}[fragile]{Basics: Plotting}

\begin{Schunk}
\begin{Sinput}
 x <- rnorm(1000)
 plot(x, xlab='x-axis', ylab='y-axis')
 hist(x, 100)
\end{Sinput}
\end{Schunk}
\begin{center}
  \includegraphics[width=\hsize]{normal}
\end{center}

\end{frame}

\begin{frame}[fragile]{Basics: linear models}
\begin{Schunk}
\begin{Sinput}
 plot(co2, ylab=expression(CO[2]~group('[', ppm, ']')))
 m <- lm(co2 ~ time(co2))
 abline(m, col=2, lwd=3); grid()
\end{Sinput}
\end{Schunk}
\vspace{-4em}
\begin{center}
  \includegraphics[width=\hsize]{co2}
\end{center}

\end{frame}

\begin{frame}[fragile]{Basics: linear models (cont'd)}
  
\begin{Schunk}
\begin{Sinput}
 summary(m)
\end{Sinput}
\begin{Soutput}
Call:
lm(formula = co2 ~ time(co2))

Residuals:
    Min      1Q  Median      3Q     Max 
-6.0399 -1.9476 -0.0017  1.9113  6.5149 

Coefficients:
              Estimate Std. Error t value Pr(>|t|)    
(Intercept) -2.250e+03  2.127e+01  -105.8   <2e-16 ***
time(co2)    1.308e+00  1.075e-02   121.6   <2e-16 ***
---
Signif. codes:  0 
\end{Soutput}
\end{Schunk}
  
\end{frame}

\section{Packages and oce}


\begin{frame}[fragile]{Packages: oce}
  
  Can install packages from CRAN with
  \verb|install.packages('oce')|. Then load the library, and use it!
  \pause
  
\begin{Schunk}
\begin{Sinput}
 library(oce)
 d <- read.oce('2008-06-25_st002.cnv')
 class(d)
\end{Sinput}
\begin{Soutput}
[1] "ctd"
attr(,"package")
[1] "oce"
\end{Soutput}
\begin{Sinput}
 str(d)
\end{Sinput}
\begin{Soutput}
Formal class 'ctd' [package "oce"] with 3 slots
  ..@ metadata     :List of 21
  .. ..$ header          : chr [1:54] "* Sea-Bird SBE 19plus Data File:" "* FileName = C:\\Documents and Settings\\Administrator\\Desktop\\SLEIWEX2008data\\2008-06-25\\seabirdCTD\\2008-06-25_st002.hex" "* Software Version Seasave Win32 V 5.31a" "* Temperature SN = 4947" ...
  .. ..$ type            : chr "SBE"
  .. ..$ hexfilename     : chr "c:\\documents and settings\\administrator\\desktop\\sleiwex2008data\\2008-06-25\\seabirdctd\\2008-06-25_st002.hex"
  .. ..$ serialNumber    : chr ""
  .. ..$ systemUploadTime: POSIXct[1:1], format: "2008-06-25 20:24:00"
  .. ..$ ship            : chr "Coriolis II"
  .. ..$ scientist       : chr "Patrick Ouellet alex hay"
  .. ..$ institute       : chr ""
  .. ..$ address         : chr ""
  .. ..$ cruise          : chr "IML2008 sleiwex2008"
  .. ..$ station         : chr "002"
  .. ..$ date            : logi NA
  .. ..$ startTime       : POSIXct[1:1], format: "2008-06-25 20:23:00"
  .. ..$ latitude        : num 47.9
  .. ..$ longitude       : num -69.8
  .. ..$ recovery        : logi NA
  .. ..$ waterDepth      : num 105
  .. ..$ sampleInterval  : logi NA
  .. ..$ names           :List of 2
  .. .. ..$ :function (x)  
  .. .. ..$ : chr "sigmaTheta"
  .. ..$ labels          :List of 2
  .. .. ..$ :function (object, ...)  
  .. .. ..$ : chr "Sigma Theta"
  .. ..$ filename        : chr "/Users/richardsc/presentations/pods2013-02-14/2008-06-25_st002.cnv"
  ..@ data         :List of 11
  .. ..$ scan         : num [1:2347] 1 2 3 4 5 6 7 8 9 10 ...
  .. ..$ time         : num [1:2347] 0 0.25 0.5 0.75 1 1.25 1.5 1.75 2 2.25 ...
  .. ..$ pressure     : num [1:2347] 0.133 0.131 0.129 0.127 0.135 0.135 0.131 0.131 0.127 0.129 ...
  .. ..$ descent      : num [1:2347] 0 -0.007 -0.007 -0.007 0.029 0 -0.015 0 -0.015 0.007 ...
  .. ..$ conductivity : num [1:2347] 0.0122 0.0122 0.0122 0.0121 0.0121 ...
  .. ..$ salinity     : num [1:2347] 0.0744 0.0742 0.074 0.0738 0.0737 0.0736 0.0734 0.0733 0.0733 0.0731 ...
  .. ..$ temperature  : num [1:2347] 14.3 14.3 14.3 14.3 14.3 ...
  .. ..$ oxygen       : num [1:2347] 7.69 7.69 7.69 7.69 7.69 ...
  .. ..$ fluorescence.: num [1:2347] 0 0 0 0 0 0 0 0 0 0 ...
  .. ..$ flag         : num [1:2347] 0 0 0 0 0 0 0 0 0 0 ...
  .. ..$ sigmaTheta   : num [1:2347] -0.743 -0.743 -0.744 -0.744 -0.744 ...
  ..@ processingLog:List of 2
  .. ..$ time : POSIXct[1:3], format: "2013-02-13 15:12:39" ...
  .. ..$ value: chr [1:3] "create 'ctd' object" "ctdAddColumn(x = res, column = swSigmaTheta(res@data$salinity,     res@data$temperature, res@data$pressure), name = \"sigmaThet"| __truncated__ "read.ctd.sbe(file = file, processingLog = processingLog)"
\end{Soutput}
\end{Schunk}

\end{frame}


\begin{frame}[fragile]
  
\begin{Schunk}
\begin{Sinput}
 summary(d)
\end{Sinput}
\begin{Soutput}
CTD Summary
-----------

* InstrumentSBE
* Instrument serial number:  
* File source:        /Users/richardsc/presentations/pods2013-02-14/2008-06-25_st002.cnv
* Original file source (hex):  c:\documents and settings\administrator\desktop\sleiwex2008data\2008-06-25\seabirdctd\2008-06-25_st002.hex
* Institute:      
* Chief scientist:      Patrick Ouellet alex hay
* Date:      NA
* Start time:          2008-06-25 20:23:00
* System upload time:  2008-06-25 20:24:00
* Cruise:              IML2008 sleiwex2008
* Vessel:              Coriolis II
* Station:             002
* Location:            47.91N  69.793W 
* Water depth:105
* Statistics of subsample::
                           Min.         Mean         Max.
    scan           1.000000e+00 1.174000e+03 2.347000e+03
    time           0.000000e+00 2.932500e+02 5.865000e+02
    pressure       1.180000e-01 2.934836e+01 9.284000e+01
    descent       -5.684000e+14 4.119472e+10 2.842200e+14
    conductivity   9.461000e-03 2.223300e+00 2.719677e+00
    salinity       5.790000e-02 2.269296e+01 3.030900e+01
    temperature    2.184800e+00 6.170377e+00 1.454720e+01
    oxygen         5.989140e+00 6.813704e+00 8.619480e+00
    fluorescence.  0.000000e+00 4.907397e+01 6.866600e+03
    flag           0.000000e+00 0.000000e+00 0.000000e+00
    sigmaTheta    -7.863291e-01 1.782229e+01 2.420303e+01
* Processing Log::
  * 2013-02-13 15:12:39 UTC: ``create 'ctd' object``
  * 2013-02-13 15:12:39 UTC: ``ctdAddColumn(x = res, column = swSigmaTheta(res@data$salinity,     res@data$temperature, res@data$pressure), name = "sigmaTheta",     label = "Sigma Theta", unit = "kg/m^3", debug = debug - 1)``
  * 2013-02-13 15:12:39 UTC: ``read.ctd.sbe(file = file, processingLog = processingLog)``
\end{Soutput}
\end{Schunk}
  
\end{frame}

\begin{frame}[fragile]
  
\begin{Schunk}
\begin{Sinput}
 plot(d)
\end{Sinput}
\end{Schunk}
\begin{center}
  \includegraphics[height=0.8\vsize, page=1]{ctd}
\end{center}

\end{frame}

\begin{frame}[fragile]
  
\begin{Schunk}
\begin{Sinput}
 plot(ctdTrim(d))
\end{Sinput}
\end{Schunk}
\begin{center}
  \includegraphics[height=0.8\vsize, page=2]{ctd}
\end{center}

\end{frame}

\begin{frame}[fragile]{Oce and other data types}
  
  From \verb|?oce|:
\begin{verbatim}
Information on the classes that derive from this base class are
     found at the following links: 'adp-class', 'adv-class',
     'cm-class', 'coastline-class', 'ctd-class', 'drifter-class',
     'echosounder-class', 'lisst-class', 'lobo-class', 'met-class',
     'sealevel-class', 'section-class', 'tdr-class', 'tidem-class',
     'topo-class', and 'windrose-class'.
\end{verbatim}
  
\vspace{-1em}
\begin{center}
  \includegraphics[width=\hsize]{adp}
\end{center}

\end{frame}

\section{Things I like}

\begin{frame}[fragile]{Things I like about R (vs Matlab)}
  
  \begin{itemize}
  \item Clean graphics (pdf output!)
    \begin{itemize}
    \item downside: not as interactive as Matlab
    \end{itemize}
  \item indexing with \verb|[...]| (e.g. \verb|x[x>5][20]|; take the
    20th value from the values of \verb|x| which are greater than 5)
  \item good help/package system
  \item named arguments in functions
  \item command line interface (\verb|$ R < script.R|)
  \item POSIX times (instead of \verb|datenum| and \verb|datetick()|)
  \end{itemize}
  
\end{frame}

\end{document}
